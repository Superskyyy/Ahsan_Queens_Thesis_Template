% Queen's Thesis Format
% (Borrowed from Dean Jin's BigDis.tex file, then heavily modified :)

% Michelle L. Crane, Queen's University, 2003

%*************************************************************************************************************
% DOCUMENT STYLE
%*************************************************************************************************************
%\documentclass[12pt]{report}
\documentclass[12pt,oneside]{report}

%%-------------------------------------------------------------------------------------------------------------
\usepackage{quthesis}        % the Queen's University dissertation style file
\usepackage[many]{tcolorbox}
\usepackage{amsmath}
\usepackage{color}
%\usepackage[table]{xcolor}
\usepackage{listings}
\usepackage{utopia} % The "Utopia" font
\usepackage{charter} % The "charter" font
\usepackage{textcomp} % Allows the use of extra characters
\usepackage{graphicx}        % for graphic images (use \includegraphics[...]{file.eps})
\usepackage[caption=false,font=footnotesize]{subfig}
\usepackage{booktabs, tabularx}        % Package used to make variable width-columns, i.e.,
\usepackage{fancybox}
\usepackage{multirow}
\usepackage{multicol}
\usepackage[linesnumbered]{algorithm2e}
\usepackage{url}
\usepackage{array}


\definecolor{mytroqous}{RGB}{220,20,60}
%\usepackage[colorlinks=true, linkcolor=mytroqous , citecolor=mytroqous , bookmarks=true,pagebackref=false,allcolors=mytroqous ]{hyperref}
\usepackage[colorlinks=true, linkcolor=blue , citecolor=blue , bookmarks=true,pagebackref=false,allcolors=blue ]{hyperref}

\usepackage[Sonny]{fncychap} % For fancy chapters
\usepackage{fancyhdr} % For fancy chapters

\usepackage{lettrine} % for large letters

\usepackage{changepage}

\usepackage{adjustbox}

%\documentclass[11pt,twoside]{report}
%\documentclass[11pt,oneside]{report}
%\usepackage[sort&compress,round]{natbib}

%\usepackage{quthesis}

%\usepackage{utopia} % The "Utopia" font
%\usepackage{charter} % The "charter" font
%\usepackage{textcomp} % Allows the use of extra characters


%\usepackage{enumerate}
%\usepackage{caption}

%\usepackage{graphicx}
%\usepackage{multirow}
%\usepackage{multicol}
%\usepackage{subfig}
%\usepackage{url}
%\usepackage{rotating}
%\usepackage{setspace}
%\usepackage{array}
%\usepackage{nth}
%\usepackage{hhline}
%\usepackage{amsmath}
%\usepackage{booktabs}
%\usepackage{ragged2e} 
%\usepackage{rotating}

%\usepackage{minted}
%\usepackage{colortbl}

%\usepackage{emptypage}
%\usepackage{tikz}
%\usetikzlibrary{arrows,automata}
%\usepackage[colorlinks=false, linkcolor=blue, citecolor=blue, bookmarks=true,pagebackref=false]{hyperref}
%\usepackage{comment}
%\usepackage{lettrine} % for large letters
%\usepackage{changepage}
%Options: Sonny, Lenny, Glenn, Conny, Rejne, Bjarne, Bjornstrup
%\usepackage[Sonny]{fncychap} % For fancy chapters
%\usepackage{fancyhdr} % For fancy chapters
%\usepackage{fancybox}
%\usepackage[colorlinks=true, linkcolor=blue, citecolor=blue, bookmarks=true,pagebackref]{hyperref}
%\usepackage{hypernat} % Need this so backreferences from hyperref (in the bibliography) work with natbibs "compress" option, i.e, [1,2,3] gets compressed to [1-3], but we need 2 to be counted

%*************************************************************************************************************
% MISCELLANEOUS COMMANDS AND ENVIRONMENTS
%*************************************************************************************************************
% Use this command to show more table of contents - used when playing
% with the draft outline
% I think it should be about 2???
\setcounter{tocdepth}{2}
%*************************************************************************************************************
% Environment definition I found in the "The Latex Companion".  Used to
% create a list environment where the indenting is the same for all of the
% entries, regardless of their length.  Note:  must \usepackage{calc}.
\newenvironment{Ventry}[1]%
    {\begin{list}{}{\renewcommand{\makelabel}[1]{\textbf{##1}\hfil}%
        \settowidth{\labelwidth}{\textbf{#1:}}%
        \setlength{\leftmargin}{\labelwidth+\labelsep}}}%
    {\end{list}}
%*************************************************************************************************************

%*************************************************************************************************************
% MY DEFINED COMMANDS
%*************************************************************************************************************
% Command that I can use to create notes in the margins;
% adapted from Juergen's META tag

\newcommand{\todo}[1]{\textcolor{blue}{\textbf{[[TODO: #1]]}}}
\newcommand{\metric}[1]{#1}
\newcommand{\className}[1]{\textbf{\sf #1}}
\newcommand{\functionName}[1]{\textbf{\sf #1}}
\newcommand{\objectName}[1]{\textbf{\sf #1}}

\newcommand{\fetch}[1]{\textbf{\em #1}}

\newcommand{\annotation}[1]{\textbf{#1}}

\newcommand{\meta}[1]{\begin{singlespacing}
{\marginpar{\emph{\footnotesize Note: #1}}}\end{singlespacing}}
%*************************************************************************************************************
% Command that I can use to create lined headings
\newcommand{\heading}[1]{\bigskip \hrule \smallskip \noindent \texttt{#1} \smallskip \hrule}
%*************************************************************************************************************
% Command that I can use for reading in a file, verbatim, with line
% numbers printed along the left side.  The parameter is the file name.
\newcommand{\fileinnum}[1]{
    \begin{singlespacing} {\footnotesize
    \begin{listinginput}[1]{1}{#1}\end{listinginput}
    }\end{singlespacing}
}
%*************************************************************************************************************
% Command that I can use for reading in a file, verbatim, with NO line
% numbers, but in a smaller font.  The parameter is the file name.
\newcommand{\filein}[1]{
    \begin{singlespacing}{\footnotesize
    \begin{verbatiminput}{#1}\end{verbatiminput}
    }\end{singlespacing}
}
%*************************************************************************************************************
% Command that I can use for reading in a file, verbatim, with NO line
% numbers, but in a smaller font.  The parameter is the file name.
\newcommand{\fileinsmall}[1]{
    \begin{singlespacing}{\scriptsize
    \begin{verbatiminput}{#1}\end{verbatiminput}
    }\end{singlespacing}
}
%*************************************************************************************************************
% Dean't 'notesbox' command.  Needs setspace package.
%   Usage: \notesbox{This is a note.}
%
\newcommand{\notesbox}[1]{
%     \ \\
      \singlespacing
      \noindent\begin{boxedminipage}[h]{\textwidth}{\sf{#1}}\end{boxedminipage}
      \doublespacing

}
\newcommand{\researchBox}[1]{
\begin{center}
\cornersize{.15} 
\setlength{\fboxsep}{8pt}
\thinlines
\ovalbox{\begin{minipage}{5.7in}
{\em #1}
\end{minipage}}

\end{center}}

\newcommand{\parhead}[1]{{\noindent \textbf{#1}}}

\newcommand{\framework}{\textsc{\em CacheOptimizer}}
\newcommand{\frameworkSpace}{\textsc{\em CacheOptimizer }}

\newcommand{\baseline}{\textsc{\em CacheAll}}
\newcommand{\baselineSpace}{\textsc{\em CacheAll }}

\newcommand{\default}{\textsc{\em DefaultCache}}
\newcommand{\defaultSpace}{\textsc{\em DefaultCache }}

\newcommand{\rowstyle}[1]{\gdef\currentrowstyle{#1}%
  #1\ignorespaces
}

\newcommand{\pattern}[1]{\emph{#1}}


\newcommand{\sid}[1]{\begin{sideways}{#1}\end{sideways}}

	% Abstract box
\newcommand{\makeAbstract}[1]{
{\small
\begin{singlespace}
\textsl{#1}
\end{singlespace}
}
}
\newcommand{\relatedPub}[1]{\vspace{1em} \noindent \textbf{Publications based on
this chapter:} \citet{#1}}

\renewcommand{\baselinestretch}{1.5}


%%%%%%%%%%%%%%%%%%%%%%%%%%%%%%
% Gray box
%%%%%%%%%%%%%%%%%%%%%%%%%%%%%%
\usepackage{fancybox}%for \hypobox
\usepackage{pstricks} 
\usepackage{balance}
\definecolor{light-gray}{gray}{0.85}	


\newcommand{\hypobox}[1]{\begin{center}%	
		\noindent\thicklines\setlength{\fboxsep}{7pt}%	
		\cornersize{0}\psframebox[framearc = 0.2, fillstyle = solid, fillcolor = light-gray]{\begin{minipage}{.95\textwidth}
				\textit{#1}
\end{minipage}} \end{center}}



\newcommand{\fancyFun}[1]{
\begin{tikzpicture}
\node [fill=cyan, rounded corners=5pt] {\large #1};
\end{tikzpicture}
}

\usepackage{multirow}
\usepackage{enumerate}
\usepackage{longtable}
\usepackage[utopia]{mathdesign}
\usepackage{helvet}
\usepackage{lscape}
\usepackage{adjustbox}
\usepackage{xcolor}
%\usepackage[numbers, sort]{natbib}
 %\usepackage[sort&compress,round]{natbib}
%\usepackage[sort&compress,round]{natbib}

\usepackage{setspace}
\usepackage{pbox}


\usepackage{framed}
\definecolor{mypink1}{rgb}{0.858, 0.188, 0.478}
\definecolor{mypink2}{RGB}{219, 48, 122}
\definecolor{mypink3}{cmyk}{0, 0.7808, 0.4429, 0.1412}
\definecolor{mygray}{gray}{0.9}

\definecolor{formalshade}{RGB}{255,255,153}    

\newenvironment{formal}{%
  \def\FrameCommand{%
    \hspace{1pt}%
    {\color{formalshade}\vrule width 6pt}%
    {\color{mygray}\vrule width 4pt}%
    \colorbox{mygray}%
  }%
  \MakeFramed{\advance\hsize-\width\FrameRestore}%
  \noindent\hspace{-4.55pt}% disable indenting first paragraph
  \begin{adjustwidth}{}{7pt}%
  \vspace{2pt}\vspace{2pt}%
}
{%
  \vspace{2pt}\end{adjustwidth}\endMakeFramed%
}


\usepackage{mathtools}

\usepackage{lscape}
 \usepackage{tabularx}
 \usepackage[figuresright]{rotating}
  
\newtcolorbox[auto counter]{summary}[1][]{title={\bfseries Summary},enhanced,
	coltitle=black,
	top=0.17in,
	attach boxed title to top left=
	{xshift=1.5em,yshift=-\tcboxedtitleheight/2},
	boxed title style={size=small,colback=lightgray},#1}
	
\newtcolorbox[auto counter]{summary_RQ1}[1][]{title={\bfseries Summary of RQ1},enhanced,
	coltitle=black,
	top=0.17in,
	attach boxed title to top left=
	{xshift=1.5em,yshift=-\tcboxedtitleheight/2},
	boxed title style={size=small,colback=lightgray},#1}

\newtcolorbox[auto counter]{summary_RQ2}[1][]{title={\bfseries Summary of RQ2},enhanced,
	coltitle=black,
	top=0.17in,
	attach boxed title to top left=
	{xshift=1.5em,yshift=-\tcboxedtitleheight/2},
	boxed title style={size=small,colback=lightgray},#1}
	
\newtcolorbox[auto counter]{summary_RQ3}[1][]{title={\bfseries Summary of RQ3},enhanced,
	coltitle=black,
	top=0.17in,
	attach boxed title to top left=
	{xshift=1.5em,yshift=-\tcboxedtitleheight/2},
	boxed title style={size=small,colback=lightgray},#1}
	
	
\newtcolorbox[auto counter]{thesis_statment}[1][]{
	coltitle=black,
	top=0.17in,
	attach boxed title to top left=
	{xshift=1.5em,yshift=-\tcboxedtitleheight/2},
	boxed title style={size=small,colback=lightgray},#1}
